\documentclass[aspectratio=1610]{beamer}
\usepackage[cm-default]{fontspec}
\setromanfont{FreeSerif}
\setsansfont{FreeSans}
\setmonofont{FreeMono}
\usepackage{geometry}
\usepackage{tikz}
\usetikzlibrary{snakes,arrows,shapes}
\usepackage{amsmath}
%\geometry{screen}
\usepackage{listings}
\lstset{
        language=Ruby,
        morekeywords={bool},
        tabsize=2,
        basicstyle={\ttfamily\tiny},
        numbers=left,
        numberstyle=\tiny,
        stepnumber=1,
        numbersep=4pt,
        showstringspaces=false,
        showspaces=false,
        showtabs=false,
        backgroundcolor=\color{white},
        commentstyle={\em\color{magenta}},
}
\usetheme{Warsaw}
\title[Provisioning Virtual Networks]{Provisioning Virtual Networks using Junos, Vagrant and Αnsible}
\author{Alexandros Kosiaris <akosiaris@gmail.com>}
\institute{GRNOG - 3rd Technical Meeting}
\date{10 Jun 2016}
\begin{document}

\AtBeginSubsection[]
{
    \begin{frame}<beamer>
        \frametitle{Layout}
        \tableofcontents[currentsection,currentsubsection]
    \end{frame}
}

\begin{frame}
    \titlepage
\end{frame}

\section{Introduction}

    \begin{frame}{Problem statement}
        Provisioning a new server/switch/router/VM has traditionally been a hard process

        \begin{itemize}
            \pause \item Time/Resource consuming
            \begin{itemize}
                \pause \item At least one person allocated to it, if not more
                \pause \item Download OS/Install OS/Reboot/Setup OS/Reboot/Reboot/Reboot
                \pause \item Fix Errors, Rinse, Repeat
            \end{itemize}
            \pause \item Causes divergence in infrastructure
            \begin{itemize}
                \pause \item People tend to do things in different ways
        \pause \item Up to  X * Y * Z possible configurations (X=\# people, Y=\# number of settings, Z=\# ways of doing something)
            \end{itemize}
        \end{itemize}

    \end{frame}

    \begin{frame}{Solution?}
    \begin{center}
    \Huge !!!Automation!!!
    \end{center}
    \end{frame}

    \begin{frame}{Solutions}
    Various approaches of automation at multiple levels:
        \begin{itemize}
        \pause \item At the image Level (Norton Ghost, Live Image, VMWare, Virtualbox, EC2, etc) - Cattle vs Pets!
        \pause \item At the installation level (Unattended Installation, Debian preseed, FAI, Kickstart, custom)
        \pause \item At the configuration level (CFEngine, Puppet, Chef, Ansible, custom) - Configuration as code!
        \end{itemize}

        \pause And their combinations whenever applicable!!!

    \end{frame}

    \begin{frame}{An informal survey}
    4 different classes (as devised by the presenter)
    \begin{itemize}
        \item No automation at all
        \item Image/Installation level automation
        \item Configuration Level automation
        \item Combination of 2 and 3
        \end{itemize}
    \end{frame}

\section{Vagrant}
    \begin{frame}{What is Vagrant?}

    \begin{block}{Wikipedia says:}
        Vagrant is computer software that creates and configures virtual development environments.
        It can be seen as a higher-level wrapper around virtualization software such as VirtualBox,
        VMware, KVM and Linux Containers (LXC), and around configuration management software such as Ansible, Chef, Salt, and Puppet.
    \end{block}

    \pause Sounds great! Now, WHAT ACTUALLY IS Vagrant?
    \end{frame}

    \begin{frame}{WHAT IS VAGRANT?}
        In the presenter's own words ?
        \begin{itemize}
        \pause \item A framework AND a tool
        \pause \item A really easy way to spin up a fleet of VMs (and/or containers)
        \pause \item A way to manage the lifecycle of VMs
        \pause \item A way to provision, ensure and manage the configuration of VMs
        \pause \item A tool to make working with systems reproducible, less buggy and faster
        \end{itemize}
    A tool to just make my (our?) lives easier
    \end{frame}

    \begin{frame}{Vagrant as a Framework}
        It's a framework. It's scaffolding and plugins, plugins, plugins
        \begin{itemize}
        \pause \item Plugins to manage VMs, called Providers (Virtualbox, VMware, EC2, Ganeti, Azure, Rackspace, Docker, you name it)
        \pause \item Plugins to provision VMs, called Provisioners (Puppet, Chef, Ansible, fabric, docker compose, you write it)
        \pause \item Plugins to "know" VMs (Linux, Windows, BSDs, CoreOS, and others)
        \pause \item Plugins to communicate with VMs (SSH, WinRM, Shared folders, etc)
        \end{itemize}
    And a growing set of things like DNS management, Host communication, snapshots and others
    \end{frame}

    \begin{frame}{Vagrant as a Tool}
        It's a framework. But it does ship with multiple plugins by default
        \begin{itemize}
        \pause \item Providers: Docker, HyperV, Virtualbox. Virtualbox is the default
        \pause \item Provisioners: Docker, Puppet (server and solo), Chef (server and solo), Ansible, CFEngine, shell.
        \end{itemize}
    \pause And allows you to download VM images off the internet
    \pause \\* And YES, Juniper does provide some to the world!
    \end{frame}

    \begin{frame}{Vagrant => Reproducible Environments}
    \begin{center}
    \Huge Key Takeaway ?
    \pause \\ REPRODUCIBLE ENVIRONMENTS!
    \end{center}
    \end{frame}

    \begin{frame}{Vagrant for the first time}
        Run \textbf{vagrant init} in an empty folder. You get the following Vagrantfile (minus the comments)
        \lstinputlisting{Vagrantfile.simple}
        Now run \textbf{vagrant up} and you have your one base VM. Run \textbf{vagrant destroy} and kill it
    \end{frame}
    \begin{frame}{Let's spice it up a bit}
        \textbf{vagrant init juniper/ffp-12.1X47-D15.4-packetmode}. Edit. It's Ruby
        \lstinputlisting{Vagrantfile.junos}
    \end{frame}

    \begin{frame}{Let's spice it up a bit (2)}
    %\enlargethispage{100cm}
    % Start of code
    % \begin{tikzpicture}[anchor=mid,>=latex',line join=bevel,]
    \begin{tikzpicture}[>=latex',line join=bevel,]
      \pgfsetlinewidth{1bp}
    %%
    \pgfsetcolor{black}
      % Edge: vsrx01 -- vsrx02
      \draw [] (67.474bp,72.765bp) .. controls (60.626bp,61.618bp) and (51.564bp,46.865bp)  .. (44.665bp,35.633bp);
      % Edge: vsrx01 -- vsrx03
      \draw [] (87.652bp,72.765bp) .. controls (94.659bp,61.618bp) and (103.93bp,46.865bp)  .. (110.99bp,35.633bp);
      % Edge: vsrx02 -- vsrx03
      \draw [] (69.111bp,18.0bp) .. controls (74.972bp,18.0bp) and (80.832bp,18.0bp)  .. (86.693bp,18.0bp);
      % Node: vsrx01
    \begin{scope}
      \definecolor{strokecol}{rgb}{0.0,0.0,0.0};
      \pgfsetstrokecolor{strokecol}
      \draw (77.447bp,90.0bp) node {vsrx01};
    \end{scope}
      % Node: vsrx03
    \begin{scope}
      \definecolor{strokecol}{rgb}{0.0,0.0,0.0};
      \pgfsetstrokecolor{strokecol}
      \draw (121.45bp,18.0bp) node {vsrx03};
    \end{scope}
      % Node: vsrx02
    \begin{scope}
      \definecolor{strokecol}{rgb}{0.0,0.0,0.0};
      \pgfsetstrokecolor{strokecol}
      \draw (34.447bp,18.0bp) node {vsrx02};
    \end{scope}
    %
    \end{tikzpicture}
    \end{frame}

\section{Ansible}
    \begin{frame}{What is Ansible?}
    \begin{block}{Wikipedia says:}
        Ansible, a free-software platform for configuring and managing computers, combines multi-node software deployment,
        ad hoc task execution, and configuration management. It manages nodes (which must have Python 2.4 or later installed on them)
        over SSH or over PowerShell. Modules work over JSON and standard output and can be written in any programming language.
        The system uses YAML to express reusable descriptions of systems.
    \end{block}
    \pause That sounds... unhelpful! WHAT ACTUALLY IS Vagrant?
    \end{frame}

    \begin{frame}{WHAT IS ANSIBLE?}
    In the presenter's own words ?
        \begin{itemize}
        \pause \item A way to write configuration
        \pause \item A way to push above said configuration to nodes
        \pause \item A way to ensure configuration of nodes
        \end{itemize}
    \begin{center}
        \pause A Configuration Management System!
    \end{center}
    \end{frame}

    \begin{frame}{How does it work?}
        So how does ansible work? The basics:
    \begin{itemize}
        \pause \item Create an inventory file
        \pause \item Define some tasks you want
        \pause \item Group tasks and hosts into a playbook
        \pause \item Apply the playbook =>  \$ ansible-playbook -i hosts apache2.yaml
    \end{itemize}
    \pause  Let's see them quickly one by one
    \end{frame}

    \begin{frame}{Ansible inventory file}
        Here's a very very simple ansible inventory.
        \lstinputlisting{ansible.inventory}
    \end{frame}
    \begin{frame}{Ansible sample tasks}
        Just install apache2 as a task
        \lstinputlisting{ansible.task}
    \end{frame}
    \begin{frame}{Ansible playbooks}
        Group them together in a playbook
        \lstinputlisting{ansible.playbook}
    \end{frame}

\section{All in the mix}
    \begin{frame}{How all this ties together?}
        \begin{itemize}
            \pause \item Vagrant provides an inventory file for us and calls ansible-playbook
            \pause \item Junos supports NETCONF
	    \pause \item Juniper provides an ansible library with various tasks for junos
            \pause \item We get to write the playbooks
        \end{itemize}
	\pause https://github.com/Juniper/ansible-junos-stdlib
    \end{frame}
    \begin{frame}{Enable RIP 1/3 - Vagrantfile}
        \lstinputlisting{Vagrantfile.junos.rip}
    \end{frame}
    \begin{frame}{Enable RIP 2/3 - rip.yaml - The playbook}
        \lstinputlisting{rip.yaml}
    \end{frame}
    \begin{frame}{Enable RIP 3/3 - rip.conf - The configuration}
        \lstinputlisting{rip.conf}
    \end{frame}
    \begin{frame}{Demo}
        \begin{center}
            \Huge DEMO time!!
        \end{center}
    \end{frame}
    \begin{frame}{Use cases}
        Where to use that thing?
        \begin{itemize}
            \pause \item Learning
            \pause \item Keeping firewall rules consistent across HA setups
            \pause \item NFV
            \pause \item You name it
        \end{itemize}
    \end{frame}
    \begin{frame}{Questions}
        \begin{center}
            \Huge Questions ?
        \end{center}
	\begin{itemize}
	    \item https://github.com/akosiaris/presentations
	    \item http://keepingitclassless.net/2015/03/go-go-gadget-networking-lab/ Kudos @ Matt Oswald
	\end{itemize}
    \end{frame}
\end{document}
