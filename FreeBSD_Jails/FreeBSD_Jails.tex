\documentclass{beamer}
\usepackage[cm-default]{fontspec}
\setromanfont{FreeSerif}
\setsansfont{FreeSans}
\setmonofont{FreeMono}
\usepackage{geometry}
%\geometry{screen}
\usepackage{listings}
\lstset{
        language=bash,
        morekeywords={bool},
        tabsize=4,
        basicstyle={\ttfamily\small},
        numbers=left,
        numberstyle=\footnotesize,
        stepnumber=1,
        numbersep=4pt,
        showstringspaces=false,
        showspaces=false,
        showtabs=false,
        backgroundcolor=\color{white},
        commentstyle={\em\color{magenta}},
}
\usetheme{Warsaw}
\title[FreeBSD Jails]{FreeBSD Jails \\ Lightweight Virtualization?}
\author{Αλέξανδρος Κοσιάρης alex@noc.grnet.gr}
\institute{Πανεπιστήμιο Πειραιά - Software Libre Society}
\date{17 Σεπτεμβρίου 2010}
\begin{document}

\AtBeginSubsection[]
{
    \begin{frame}<beamer>
        \frametitle{Layout}
        \tableofcontents[currentsection,currentsubsection]
    \end{frame}
}

\begin{frame}
    \titlepage
\end{frame}

\section{Εισαγωγή}

    \begin{frame}{Το πρόβλημα}
        Το κλασσικό μοντέλο ασφαλείας (DAC) του UNIX κάνει δύσκολη τη ζωή του διαχειριστή

        \begin{itemize}
            \pause \item Ανθρώποι/ομάδες που δεν έχουν σχέσεις εμπιστοσύνης μεταξύ τους (ή ακόμη και με τον ίδιο)
            \pause \item Συνεχής ανάγκη αναπροσαρμογής των κανόνων από τον διαχειριστή
            \pause \item Λιγοστές δυνατότητες ανάθεσης δικαιωμάτων σε άλλους χρήστες
            \pause \item Πρακτικά δύο είδη χρηστών: ο root, και... οι υπόλοιποι
        \end{itemize}

    \end{frame}

    \begin{frame}{Λύσεις;}
        Έχουν προταθεί και υλοποιηθεί διάφορες λύσεις

        \begin{itemize}
            \item \texttt{sudo}
            \item \texttt{chroot(2)}
            \item Filesystem ACLs
            \item Mandatory Access Control (SELinux, AppArmor, TrustedBSD, RSBAC, Solaris roles,...)
            \item Virtualization...
        \end{itemize}

        \pause Κάποια ασχολούνται με να κλειδώσουν κάπου(και σε κάποια επίπεδα) τους χρήστες δίνοντας τους μόνο τα απαραίτητα, άλλα προσθέτουν δυνατότητες στο DAC ώστε να είναι πιο εύκολος ο έλεγχος και κάποια επιλέγουν να τους πετάξουν σε ένα άλλο παράλληλο και μακρινό σύμπαν

    \end{frame}

\section{FreeBSD Jails}
    \begin{frame}{Ακόμη μία λύση;}

        \begin{block}{Ακόμη μία λύση!}
            Enter FreeBSD Jails
        \end{block}

        Η ιδέα προτάθηκε από τον Poul-Henning Kamp και τον Robert Watson το 2000, και εμφανίστηκε για πρώτη φορά(υλοποιημένη από τον phk) στο FreeBSD 4.0

    \end{frame}

    \begin{frame}{Τι είναι τα Jails;}
        \begin{block}{FreeBSD jail είναι:}
            Μία εξέλιξη του chroot(2) syscall ώστε να υπάρχει διαχωρισμός μίας διεργασίας και των παιδιών της όχι μόνο στο επίπεδο του filesystem αλλά και στα επίπεδα των διεργασιών και δικτύου
        \end{block}
        Μία διεργασία μέσα σε ένα jailed περιβάλλον δεν μπορεί:
        \begin{itemize}
            \item Να δεί ή αλληλεπιδράσει με διεργασίες εκτός του jail
            \item Να μεταβάλλει με οποιονδήποτε τρόπο τον πυρήνα
            \item Να προσαρτήσει(mount) άλλα filesystems
            \item Να δει άλλο μέρος του filesystem πέρα του δικού του
            \item Να δημιουργήσει device nodes(aka mknod)
            \item Nα αλλάξει/χρησιμοποιήσει δικτυακούς πόρους
        \end{itemize}

    \end{frame}

    \begin{frame}{Απαιτούμενα}
        Ένα jail χρειάζεται:
        \begin{itemize}
            \item Ένα FreeBSD filesystem hierarchy
            \item Μία IP διεύθυνση(μοιρασμένη με τον host)(optional)
            \item Μία διεργασία που θα τρέξει στoν jailed χώρο
        \end{itemize}
        Εάν το αρχικό πρόγραμμα που θα τρέξει είναι το /etc/rc μπορούμε να σηκώσουμε ένα ολόκληρο Virtual FreeBSD σύστημα

    \end{frame}

    \begin{frame}{Πλεονεκτήματα}
        \begin{itemize}
            \item Εξαιρετικά φτηνό virtualization - Ελάχιστη χρήση σε δίσκο και κυρίως μνήμη
            \pause \item Πλήρης διαχωρισμός των διαφόρων jails - χρηστών/ομάδων
            \pause \item Προσβάσιμο από τον host το guest σύστημα 
            \begin{itemize}
                \item Διαχείριση
                \item Αναβάθμιση
                \item Παρακολούθηση
            \end{itemize}
        \end{itemize}
    \end{frame}

    \begin{frame}{Μειονεκτήματα}
        \begin{itemize}
        \item Κοινό networking stack για όλα τα jails (και τον host)
        \pause \item Έλλειψη εύκολα διαχειρίσιμων μηχανισμών quota
        \begin{itemize}
            \pause \item Δίσκου
            \pause \item Μνήμης
            \pause \item CPU (Τα CPU Sets διορθώνουν αυτό μερικώς στο FreeBSD 7.1)
            \pause \item Δικτύου
        \end{itemize}
        \pause \item Λειψό networking stack
            \begin{itemize}
                \pause \item Μία IPv4 διεύθυνση per jail ( για <FreeBSD 8.0)
                \pause \item No IPv6 ( για < FreeBSD 8.0)
                \pause \item To localhost είναι βραχυκυκλωμένο. Προβλήματα με TCP wrappers,
                             Application Level ACLs(πχ apache)
                \pause \item No firewall support
            \end{itemize}
        \end{itemize}
    \end{frame}

\section{Low-level διαδικασία κατασκευής jail}

    \begin{frame}{Προετοιμασία Host}
    \begin{itemize}
        \item Bind όλες οι υπηρεσίες στην μία IP,IPv6 του host
        \begin{itemize}
        \item sendmail\_enable="NO"
        \item inetd\_flags="-wW -a 10.10.10.10"
        \item rpcbind\_enable="NO"
        \item syslogd\_flags="-s -b 10.10.10.10"
        \item sshd\_config ListenAddress
        \end{itemize}
    \end{itemize}
    \end{frame}

    \begin{frame}{Προετοιμασία Host (2)}
        \lstinputlisting{raw.sh}
    Εργαλεία
    \begin{itemize}
        \item jls - Δείχνει τα running jails
        \item jexec - Με argument το jailid (από την jls) και ένα command(πχ tcsh) μπαίνεις στο jail
    \end{itemize}
    \end{frame}

    \begin{frame}{Προετοιμασία Jail}
    Ένα νέο jail είναι σαν να έχει γίνει μόλις installed. Χρειάζεται:
    \begin{itemize}
        \item Ένα άδειο fstab
        \item Disabled τον portmapper (rpcbind\_enable="NO" στο /etc/rc.conf")
        \item /etc/resolv.conf
        \item newaliases για να μην γκρινιάζει το sendmail
        \item network\_interfaces="" στο /etc/rc.conf
        \item Root password
        \item Timezone
    \end{itemize}
    Και είμαστε έτοιμοι.
    \end{frame}

\section{Tips and tricks}
    \begin{frame}
            \frametitle{OS Tricks}
        Για να ξεκινάνε τα jails on host boot βάζουμε περίπου τα εξής στο /etc/rc.conf
        \lstinputlisting{rc.conf}
    \end{frame}

    \begin{frame}{jailctl}
        Σχετικά παλιό εργαλείο. Επιτρέπει εύκολα:
        \begin{itemize}
            \item Δημιουργία/καταστροφή
            \item Εκκίνηση/τερματισμό
            \item Αναβάθμιση/backup/restore
        \end{itemize}
        Μειονεκτήματα:
        \begin{itemize}
            \item Το πρώτο jail φτιάχνεται χειροκίνητα
            \item Σχετικά υψηλή χρήση δίσκου(σε σχέση με άλλα εργαλεία)
            \item Non-maintained
        \end{itemize}
    \end{frame}

    \begin{frame}{ezjail}
        Νεώτερο εργαλείο. Ακόμη αναπτύσσεται. Επιτρέπει:
        \begin{itemize}
            \item Binary εγκατάσταση/αναβάθμιση/καταστροφή
            \item Πολύ φτηνά jails σε χώρο(2ΜΒ)
            \item Εύκολη αναβάθμιση
            \item Flavours(πχ FAMP, LDAP κτλ)
            \item ZFS jails
            \item Jail specific Routing Tables ( μέσω του setfib(2) )
            \item CPUSets
        \end{itemize}
        Ένα μειονέκτημα: Δεν ξεκινά αυτόματα τα jails με δικτύωση. Υπάρχουν trivial patches αλλά δεν γίνονται δεκτά upstream. Ευτυχώς το patch management των FreeBSD ports δουλεύει υπέροχα.
    \end{frame}

\section{Back to Future}
    \begin{frame}{FreeBSD 8.x}
        \begin{block}{VIMAGE}
            VIMAGE(Experimental): Ένας νέος τύπος container. Πρόκειται για ένα jail με virtualized networking stack
        \end{block}
        Virtualized:
        \begin{itemize}
            \item IPv4,IPv6
            \item NFS
            \item IPFW / PF firewalls,
            \item BPF, raw / routing sockets...
        \end{itemize}
    \end{frame}

    \begin{frame}{Use Cases}
        Χρησιμοποιείται από:
        \begin{itemize}
            \item PostgreSQL
            \item eUKhost
            \item RootBSD
            \item InetServices
            \item NTUA NOC
        \end{itemize}
    \end{frame}

\end{document}
